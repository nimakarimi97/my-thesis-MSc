% !TEX root = ../my-thesis.tex
%
\pdfbookmark[0]{Abstract}{Abstract}
\addchap*{Abstract}
\label{sec:abstract}

This thesis presents a comprehensive study on developing a machine learning model to predict mobility patterns. The primary focus of this project is to utilize GPS data and assess the effectiveness of machine learning models in forecasting whether users use a lift on their way or not considering route characteristics. Additionally, the research aims to identify the key factors influencing mobility behavior and explore the potential application of machine learning models in personalized mobility prediction and sustainable transportation planning.

To achieve these objectives, a range of machine learning models will be evaluated for their accuracy and precision in forecasting mobility patterns. The research will investigate how different models perform in predicting mobility transportation. Moreover, an in-depth analysis will be conducted to identify the factors that significantly impact the accuracy of these models.

By examining the accuracy and precision of various machine learning models, this research will provide valuable insights into their effectiveness in mobility forecasting. Furthermore, it will uncover the factors that play a crucial role in influencing the accuracy of these models. The findings of this study will contribute to the advancement of machine learning applications in transportation planning and assist in developing personalized mobility prediction systems for sustainable transportation.

Keywords: machine learning, mobility patterns, accuracy, precision, personalized mobility prediction, sustainable transportation planning

\newpage

{\usekomafont{chapter}Riassunto}
\label{sec:abstract-it}

Questa tesi presenta uno studio completo sullo sviluppo di un modello di apprendimento automatico per prevedere i modelli di mobilità. L'obiettivo principale di questo progetto è utilizzare i dati GPS e valutare l'efficacia dei modelli di apprendimento automatico nel prevedere se gli utenti utilizzeranno un ascensore lungo il percorso o non considereranno le caratteristiche del percorso. Inoltre, la ricerca mira a identificare i fattori chiave che influenzano il comportamento di mobilità ed esplorare la potenziale applicazione dei modelli di apprendimento automatico nella previsione personalizzata della mobilità e nella pianificazione dei trasporti sostenibili.

Per raggiungere questi obiettivi, una serie di modelli di apprendimento automatico verranno valutati per la loro accuratezza e precisione nella previsione dei modelli di mobilità. La ricerca esaminerà le prestazioni dei diversi modelli nel prevedere la mobilità dei trasporti. Inoltre, verrà condotta un’analisi approfondita per identificare i fattori che incidono in modo significativo sull’accuratezza di questi modelli.

Esaminando l’accuratezza e la precisione di vari modelli di apprendimento automatico, questa ricerca fornirà preziose informazioni sulla loro efficacia nella previsione della mobilità. Inoltre, scoprirà i fattori che svolgono un ruolo cruciale nell’influenzare l’accuratezza di questi modelli. I risultati di questo studio contribuiranno al progresso delle applicazioni di apprendimento automatico nella pianificazione dei trasporti e aiuteranno nello sviluppo di sistemi di previsione della mobilità personalizzati per il trasporto sostenibile.

Parole chiave: machine learning, modelli di mobilità, accuratezza, precisione, previsione personalizzata della mobilità, pianificazione dei trasporti sostenibili

