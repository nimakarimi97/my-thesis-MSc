% !TEX root = ../my-thesis.tex
%
\chapter{Results}
\label{sec:results}

As discussed in \ref{sec:overfit}, the model architecture should be neither too simple to be underfitted and nor too complex to be overfitted. 
In this case, the dataset and the relationship between the dependent variables and the independent variable were pretty straightforward for the models to predict the essence of lifts. 
The dataset that is relevant to the problem being solved ensures that the model learns meaningful patterns and relationships, enhancing accuracy
\cite{gudivada2017data}.

Note that the dataset has high precision and accurate values and the fact that the minimal presence of null values further contributed to desirable output. With that being said, even though cross validation has been used to prohibit any sort of overfitting, the results were still satisfying.
It is imperative to divide data into training, validation, and testing sets to fairly evaluate the model's accuracy on unseen data. Moreover, note that a larger dataset is preferred as it allows the model to learn diverse patterns and reduces the risk of overfitting, ultimately leading to better accuracy.





\section{Section 1}
\label{sec:results:sec1}


* put confusion\_matrix figure




\section{Section 2}
\label{sec:results:sec3}



\section{Conclusion}
\label{sec:results:conclusion}




\section{Future Work}
\label{sec:results:future}
